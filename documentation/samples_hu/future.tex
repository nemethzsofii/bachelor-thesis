\chapter{Továbbfejlesztési lehetőségek}
\label{ch:future}

Egy alkalmazást véleményem szerint gyakorlatilag soha nem lehet teljesen befejezettnek tekinteni. Mindig lehet hova fejleszteni, mindig lesznek új ötletek hozzá, illetve ha más nem is, újabb technológiák egészen biztosan. És egy programozó is folyamatosan tanul, és egyre jobb és jobb termékeket igyekszik kiadni a kezei közül.

Ezen webalkalmazás továbbfejlesztésére sok ötletem van, amik megvalósítása már nem esett jelen projekt keretei közé. A következő fejezetben ezeket fogom röviden kifejteni.

\section{Statisztikák}
A "Reports" oldal jelenleg 2 egyszerűen kiszámított diagramot tartalmaz csak, de itt azt gondolom sokkal nagyobb potenciál is lehetne. Akár AI modellekkel kiszámolni a statisztikákat, a költési trendeket és ezáltal személyre szabott tanácsadást biztosítani a felhasználónak. statisztika sokszínű eszközeit kihasználva az adattengerből hasznos következtetéseket levonni.
\section{Tranzakciók}
A tranzakciók lekövetésére és megjelenítésére is számos jobb módszer lehetne, mint amit én ebben a projektben megvalósítottam. A felhasználónak például jelentősen megkönnyítené a dolgát, ha nem manuálisan kéne bevinnie az adott költéseket, hanem blokkokat fényképezve, vagy akár bankártyához csatolva lehetne nyomon követni ezeket.
\section{Csoportok}
A csoportos funkciókban rengeteg potenciál bújkál még. Közös megtakarítások, motivációs taktikák.
\section{UI}
A felhasználói élmény szempontjából kiemelten fontos, hogy egy alkalmazás felülete átlátható, letisztult és esztétikus legyen. A modern webes elvárások világában a vanilla HTML, CSS és JavaScript (még Bootstrap-tel kiegészítve is) már nem feltétlenül elegendő ahhoz, hogy igazán igényes, vizuálisan is kiemelkedő felhasználói felületet hozzunk létre. Emellett például egy React vagy más frontend keretrendszer alkalmazása nemcsak látványosabb UI-t tesz lehetővé, de strukturáltabb, könnyebben karbantartható kódot is eredményez. Ezért érdemes lehet a felhasználói felületet egy modern frameworkben újraalkotni.