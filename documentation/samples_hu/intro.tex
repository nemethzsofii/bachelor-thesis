\chapter{Bevezetés}
\label{ch:intro}

A választott témám egy személyes pénzügyeket rendszerező webalkalmazás. A dolgozat a full-stack webfejlesztés témaköréhez kapcsolódik, mivel a frontend, backend és adatbázis kezelést is magába foglalja. A backendet egy C\# alapú ASP.NET Web App (Razor Pages) keretrendszerrel valósítottam meg, melyhez egy MySQL relációs adatbázist csatoltam. A frontend részt pedig egy egyszerű HTML/CSS/JS (JavaScript) kombináció alkotja, Bootstrap (CSS Framework) elemeket felhasználva.

Az alkalmazás fő lényege, hogy költéseinket és bevételeinket tudjuk nyomon követni, illetve spórolási segítséget nyújt, kimutatásokat készít a pénzkezelési szokásainkról, mindezt egy kifinomult, egyszerű és felhasználóbarát felületen keresztül. A jelenlegi megoldások erre a célra véleményem szerint nem elég egyszerűek, vagy pedig túl drágák egy hétköznapi ember számára.

A következő fejezetekben fogom kifejteni a projektet bővebben a felhasználói és fejlesztői dokumentációkon keresztül.
