\chapter{Bevezetés}
\label{ch:intro}

A választott témám egy személyes pénzügyeket rendszerező webalkalmazás. A dolgozat a full-stack webfejlesztés témaköréhez kapcsolódik, mivel frontend-, backend- és adatbáziskezelést is magában foglal. A backendet és a frontendet egy projekt részeként, egy C\# alapú ASP.NET Web App (Razor Pages) (.NET 8) keretrendszerrel valósítottam meg, amelyhez egy MySQL relációs adatbázist csatoltam az Entity Framework segítségével. A .NET lehetőséget nyújt különböző frontend keretrendszerek használatára is, azonban a webalkalmazásomat jelenleg egy egyszerű HTML/CSS/JavaScript kombináció alkotja, Bootstrap (CSS framework) elemek felhasználásával.

Az alkalmazás fő lényege, hogy a felhasználó költéseit és bevételeit nyomon tudja követni, megtakarításait kezelhesse akár egyénileg, akár csoportokban, mindezt egy felhasználóbarát felületen. Az alkalmazás az alap funkciókon kívül még kimutatásokat is készít a felhasználó pénzkezelési szokásairól, melyet akár PDF formátumban is le lehet tölteni.


Véleményem szerint a már erre a célre létrehozott megoldások egy része túlságosan leegyszerűsített, másik része pedig annyira összetett és funkciógazdag, hogy hétköznapi felhasználók számára nehezen átláthatóvá válik.


A következő fejezetekben részletesen bemutatom a projektet, felhasználói és fejlesztői dokumentációk segítségével.
