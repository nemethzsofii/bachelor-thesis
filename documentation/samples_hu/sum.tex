\chapter{Összegzés}
\label{ch:sum}

Összegezve tehát az alkalmazás elsősorban a spórolást, tudatos pénzköltést szeretné promótálni, és erre kínál magánszemélyeknek vagy háztartásoknak/családoknak egy hasznos eszközt. A fő funkció a költéseink és bevételeink vezetése, amely az első lépés a tudatosság és a pénzügyi szabadság irányába. Ezután a következő lépésnek tekinthetjük azt, amikor már az okosabb költekezés következményeképpen többlet bevételünk is lesz, melyet elkezdünk félrerakni. A megtakarításainkat is nyomon tudjuk követni az alkalmazáson belül, sőt még kimutatásokat is láthatunk, melyek rendszeres tanulmányozásával különböző mintázatokat vélhetünk felfedezni pénzügyi szokásainkból, amelyekből a tanulságokat levonva, tovább tudjuk optimalizálni az anyagiak kezelését.

A webalkalmazást egy C\# alapú keretrendszerrel (ASP.NET Web App, Razor Pages) valósítottam meg, amely egy adatbázissal is természetesen összeköttetésben áll. A fejlesztés során végig törekedtem a szabványos és jól strukturált felépítésre, a Clean Code elveinek, valamint az adatbázistervezés bevált gyakorlatainak követésére. A felhasználói felület modern és letisztult megjelenést kapott, így nem csupán az élményre helyeztem a hangsúlyt, hanem az alkalmazás jövőbeni továbbfejleszthetőségére, a kód olvashatóságára és könnyű karbantarthatóságára is.
